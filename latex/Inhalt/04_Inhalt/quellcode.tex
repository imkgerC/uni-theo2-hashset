\chapter{Programm- bzw. Quellcode}
\section{Quellcode}
Der komplette Quellcode inklusive Graph-Generierung wurde in Rust geschrieben. Die Ausführung sollte über das Tool \enquote{cargo} durchgeführt werden. Das erhaltene Executable gibt die gemessenen Daten in Tabellenform in stdout aus, erstellt eine csv-Datei mit den Daten und erstellt 4 Graphen. Die Konfiguration der Messung erfolgt über den Quellcode.

Da gefordert wurde, dass der komplette Quellcode im Dokument enthalten ist, hier eine Sektion mit demselben. Alternativ kann das Projekt auch angenehm auf GitHub unter \footnote{https://github.com/imkgerC/uni-theo2-hashset} eingesehen werden. Die Dokumentation und Erklärung des Codes ist über \enquote{Doc Comments} realisiert, kann also im Code durch Kommentare komplett eingebettet gelesen werden. Alternativ kann auch über den Aufruf des Befehls \code{cargo doc} eine HTML-Dokumentation generiert werden.
\section{Vollständiger Quellcode}
\lstinputlisting[label=code:main, caption=main.rs, language=Rust, style=colouredRust]{../src/main.rs}
\lstinputlisting[label=code:mod, caption=mod.rs, language=Rust, style=colouredRust]{../src/hashset/mod.rs}
\lstinputlisting[label=code:hashing, caption=hashing.rs, language=Rust, style=colouredRust]{../src/hashset/hashing.rs}
\lstinputlisting[label=code:probing, caption=probing.rs, language=Rust, style=colouredRust]{../src/hashset/probing.rs}
\lstinputlisting[label=code:chainingtable, caption=chainingtable.rs, language=Rust, style=colouredRust]{../src/hashset/chainingtable.rs}
\lstinputlisting[label=code:openaddressing, caption=openaddressing.rs, language=Rust, style=colouredRust]{../src/hashset/openaddressing.rs}
\lstinputlisting[label=code:coalescedtable, caption=coalescedtable.rs, language=Rust, style=colouredRust]{../src/hashset/coalescedtable.rs}
\lstinputlisting[label=code:logging, caption=logging.rs, language=Rust, style=colouredRust]{../src/logging.rs}
